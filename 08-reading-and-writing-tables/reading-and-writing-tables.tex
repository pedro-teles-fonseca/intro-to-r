\PassOptionsToPackage{unicode=true}{hyperref} % options for packages loaded elsewhere
\PassOptionsToPackage{hyphens}{url}
\PassOptionsToPackage{dvipsnames,svgnames*,x11names*}{xcolor}
%
\documentclass[ignorenonframetext,]{beamer}
\usepackage{pgfpages}
\setbeamertemplate{caption}[numbered]
\setbeamertemplate{caption label separator}{: }
\setbeamercolor{caption name}{fg=normal text.fg}
\beamertemplatenavigationsymbolsempty
% Prevent slide breaks in the middle of a paragraph:
\widowpenalties 1 10000
\raggedbottom
\setbeamertemplate{part page}{
\centering
\begin{beamercolorbox}[sep=16pt,center]{part title}
  \usebeamerfont{part title}\insertpart\par
\end{beamercolorbox}
}
\setbeamertemplate{section page}{
\centering
\begin{beamercolorbox}[sep=12pt,center]{part title}
  \usebeamerfont{section title}\insertsection\par
\end{beamercolorbox}
}
\setbeamertemplate{subsection page}{
\centering
\begin{beamercolorbox}[sep=8pt,center]{part title}
  \usebeamerfont{subsection title}\insertsubsection\par
\end{beamercolorbox}
}
\AtBeginPart{
  \frame{\partpage}
}
\AtBeginSection{
  \ifbibliography
  \else
    \frame{\sectionpage}
  \fi
}
\AtBeginSubsection{
  \frame{\subsectionpage}
}
\usepackage{lmodern}
\usepackage{amssymb,amsmath}
\usepackage{ifxetex,ifluatex}
\usepackage{fixltx2e} % provides \textsubscript
\ifnum 0\ifxetex 1\fi\ifluatex 1\fi=0 % if pdftex
  \usepackage[T1]{fontenc}
  \usepackage[utf8]{inputenc}
  \usepackage{textcomp} % provides euro and other symbols
\else % if luatex or xelatex
  \usepackage{unicode-math}
  \defaultfontfeatures{Ligatures=TeX,Scale=MatchLowercase}
\fi
% use upquote if available, for straight quotes in verbatim environments
\IfFileExists{upquote.sty}{\usepackage{upquote}}{}
% use microtype if available
\IfFileExists{microtype.sty}{%
\usepackage[]{microtype}
\UseMicrotypeSet[protrusion]{basicmath} % disable protrusion for tt fonts
}{}
\IfFileExists{parskip.sty}{%
\usepackage{parskip}
}{% else
\setlength{\parindent}{0pt}
\setlength{\parskip}{6pt plus 2pt minus 1pt}
}
\usepackage{xcolor}
\usepackage{hyperref}
\hypersetup{
            pdftitle={Introduction to R Programming},
            pdfauthor={Pedro Fonseca},
            colorlinks=true,
            linkcolor=Maroon,
            filecolor=Maroon,
            citecolor=Blue,
            urlcolor=blue,
            breaklinks=true}
\urlstyle{same}  % don't use monospace font for urls
\newif\ifbibliography
\usepackage{color}
\usepackage{fancyvrb}
\newcommand{\VerbBar}{|}
\newcommand{\VERB}{\Verb[commandchars=\\\{\}]}
\DefineVerbatimEnvironment{Highlighting}{Verbatim}{commandchars=\\\{\}}
% Add ',fontsize=\small' for more characters per line
\usepackage{framed}
\definecolor{shadecolor}{RGB}{248,248,248}
\newenvironment{Shaded}{\begin{snugshade}}{\end{snugshade}}
\newcommand{\AlertTok}[1]{\textcolor[rgb]{0.94,0.16,0.16}{#1}}
\newcommand{\AnnotationTok}[1]{\textcolor[rgb]{0.56,0.35,0.01}{\textbf{\textit{#1}}}}
\newcommand{\AttributeTok}[1]{\textcolor[rgb]{0.77,0.63,0.00}{#1}}
\newcommand{\BaseNTok}[1]{\textcolor[rgb]{0.00,0.00,0.81}{#1}}
\newcommand{\BuiltInTok}[1]{#1}
\newcommand{\CharTok}[1]{\textcolor[rgb]{0.31,0.60,0.02}{#1}}
\newcommand{\CommentTok}[1]{\textcolor[rgb]{0.56,0.35,0.01}{\textit{#1}}}
\newcommand{\CommentVarTok}[1]{\textcolor[rgb]{0.56,0.35,0.01}{\textbf{\textit{#1}}}}
\newcommand{\ConstantTok}[1]{\textcolor[rgb]{0.00,0.00,0.00}{#1}}
\newcommand{\ControlFlowTok}[1]{\textcolor[rgb]{0.13,0.29,0.53}{\textbf{#1}}}
\newcommand{\DataTypeTok}[1]{\textcolor[rgb]{0.13,0.29,0.53}{#1}}
\newcommand{\DecValTok}[1]{\textcolor[rgb]{0.00,0.00,0.81}{#1}}
\newcommand{\DocumentationTok}[1]{\textcolor[rgb]{0.56,0.35,0.01}{\textbf{\textit{#1}}}}
\newcommand{\ErrorTok}[1]{\textcolor[rgb]{0.64,0.00,0.00}{\textbf{#1}}}
\newcommand{\ExtensionTok}[1]{#1}
\newcommand{\FloatTok}[1]{\textcolor[rgb]{0.00,0.00,0.81}{#1}}
\newcommand{\FunctionTok}[1]{\textcolor[rgb]{0.00,0.00,0.00}{#1}}
\newcommand{\ImportTok}[1]{#1}
\newcommand{\InformationTok}[1]{\textcolor[rgb]{0.56,0.35,0.01}{\textbf{\textit{#1}}}}
\newcommand{\KeywordTok}[1]{\textcolor[rgb]{0.13,0.29,0.53}{\textbf{#1}}}
\newcommand{\NormalTok}[1]{#1}
\newcommand{\OperatorTok}[1]{\textcolor[rgb]{0.81,0.36,0.00}{\textbf{#1}}}
\newcommand{\OtherTok}[1]{\textcolor[rgb]{0.56,0.35,0.01}{#1}}
\newcommand{\PreprocessorTok}[1]{\textcolor[rgb]{0.56,0.35,0.01}{\textit{#1}}}
\newcommand{\RegionMarkerTok}[1]{#1}
\newcommand{\SpecialCharTok}[1]{\textcolor[rgb]{0.00,0.00,0.00}{#1}}
\newcommand{\SpecialStringTok}[1]{\textcolor[rgb]{0.31,0.60,0.02}{#1}}
\newcommand{\StringTok}[1]{\textcolor[rgb]{0.31,0.60,0.02}{#1}}
\newcommand{\VariableTok}[1]{\textcolor[rgb]{0.00,0.00,0.00}{#1}}
\newcommand{\VerbatimStringTok}[1]{\textcolor[rgb]{0.31,0.60,0.02}{#1}}
\newcommand{\WarningTok}[1]{\textcolor[rgb]{0.56,0.35,0.01}{\textbf{\textit{#1}}}}
\setlength{\emergencystretch}{3em}  % prevent overfull lines
\providecommand{\tightlist}{%
  \setlength{\itemsep}{0pt}\setlength{\parskip}{0pt}}
\setcounter{secnumdepth}{0}

% set default figure placement to htbp
\makeatletter
\def\fps@figure{htbp}
\makeatother


\title{Introduction to R Programming}
\providecommand{\subtitle}[1]{}
\subtitle{Reading and Writing Tables}
\author{Pedro Fonseca}
\date{20 Abril 2020}

\begin{document}
\frame{\titlepage}

\begin{frame}[fragile]{Before importing data}
\protect\hypertarget{before-importing-data}{}

\begin{itemize}
\item
  Are you in the correct working directory? Are you in a project?
\item
  Remember \texttt{getwd()} and \texttt{setwd()}.
\item
  Typically we import tables stored in text files or CSV files (comma
  separated values).
\item
  There are packages that allow import xls and xlsx tables into R, but
  if you want to read data from excel I reccomend saving your excel
  worksheet as a CSV or text (tab delimited) file instead.
\item
  It is important to open the file were the table is stored with a text
  editor and pay attention to the structure of the data.
\end{itemize}

\end{frame}

\begin{frame}[fragile]{The source file}
\protect\hypertarget{the-source-file}{}

What to look for:

\begin{itemize}
\item
  Headers (column names)
\item
  Separator (\texttt{","}, \texttt{";"}, ``\textbackslash{}t'',
  \ldots{})
\item
  How are strings represented? (with \texttt{"\textbackslash{}""} ?)
\item
  Decimal separators (\texttt{"."}, or \texttt{","}?)
\item
  Row names
\item
  How are \texttt{NA}s represented? (\texttt{"\ "}, \texttt{"NA"},
  \texttt{"na"}, \texttt{"Na"}, \texttt{"."}, \texttt{"-"}, \ldots{} )
\end{itemize}

\end{frame}

\begin{frame}{The source file}
\protect\hypertarget{the-source-file-1}{}

What to look for:

\begin{itemize}
\item
  Should we read all the lines of the file?
\item
  Is there any metadata? Does the table start in the first line of the
  file?
\item
  Are there comments? What is the comment character?
\end{itemize}

\end{frame}

\begin{frame}[fragile]{Base R functions for reading tables}
\protect\hypertarget{base-r-functions-for-reading-tables}{}

\begin{itemize}
\tightlist
\item
  All we need is \texttt{read.table()}!
\end{itemize}

But there are some useful wrappers around \texttt{read.table()}:

\begin{itemize}
\tightlist
\item
  \texttt{read.csv()}
\item
  \texttt{read.csv2()}
\item
  \texttt{read.delim()}
\item
  \texttt{read.delim2()}
\end{itemize}

These functions read a file in table format and create a data frame that
you can assign to an R object.

\end{frame}

\begin{frame}[fragile]{\texttt{read.table()}}
\protect\hypertarget{read.table}{}

\texttt{read.table()} is our powerhouse function for reading tables from
external files These are the most important defaults:

\begin{Shaded}
\begin{Highlighting}[]
\KeywordTok{read.table}\NormalTok{(file, }\DataTypeTok{header =} \OtherTok{FALSE}\NormalTok{, }\DataTypeTok{sep =} \StringTok{""}\NormalTok{, }
           \DataTypeTok{quote =} \StringTok{"}\CharTok{\textbackslash{}"}\StringTok{'"}\NormalTok{, }\DataTypeTok{dec =} \StringTok{"."}\NormalTok{, row.names,}
\NormalTok{           col.names, }\DataTypeTok{na.strings =} \StringTok{"NA"}\NormalTok{, }
           \DataTypeTok{nrows =} \DecValTok{-1}\NormalTok{, }\DataTypeTok{skip =} \DecValTok{0}\NormalTok{, }\DataTypeTok{comment.char =} \StringTok{"#"}\NormalTok{)}
\end{Highlighting}
\end{Shaded}

\begin{itemize}
\tightlist
\item
  The \texttt{file} argument is where you specify the name of the file
  you want to read.
\item
  If the file is not in the working directory you should indicate the
  path to the file instead.
\item
  File names and paths must be provided as strings of characters.
\end{itemize}

\end{frame}

\begin{frame}[fragile]{\texttt{read.table()}}
\protect\hypertarget{read.table-1}{}

\begin{itemize}
\item
  To the \texttt{row.names} argument you can either provide a vector
  with the names or a single number or character string indicating the
  position or the name of the column containing the names.
\item
  If there is a header and the first row contains one fewer field than
  the number of columns, the first column is used for the row names. If
  \texttt{row.names} is missing, the rows are numbered.
\end{itemize}

\end{frame}

\begin{frame}[fragile]{Wrappers around \texttt{read.table()}}
\protect\hypertarget{wrappers-around-read.table}{}

\texttt{read.csv()} and \texttt{read.csv2()} are identical to
\texttt{read.table()} except for the defaults. They are intended for
reading CSV files with:

\begin{itemize}
\item
  \texttt{read.csv()}: \texttt{sep\ =\ ","} and \texttt{dec\ =\ "."}
\item
  \texttt{read.csv2()}: \texttt{sep\ =\ ";"} and \texttt{dec\ =\ ","}
\end{itemize}

Also, both \texttt{read.csv()} and \texttt{read.csv2()} have
\texttt{header\ =\ TRUE} and and \texttt{comment.char\ =\ ""}, which
means that the comment character is disabled.

\end{frame}

\begin{frame}[fragile]{Wrappers around \texttt{read.table()}}
\protect\hypertarget{wrappers-around-read.table-1}{}

\texttt{read.delim()} and \texttt{read.delim2()}are identical to
\texttt{read.table()} except for the defaults. They are intended for
reading TAB delimited files (sep = ``\textbackslash{}t'') with:

\begin{itemize}
\tightlist
\item
  \texttt{read.delim()}: dec = ``.''
\item
  \texttt{read.delim2()}: dec = ``,''
\end{itemize}

Also, both \texttt{read.delim()} and \texttt{read.delim2()} have
\texttt{header\ =\ TRUE} and and \texttt{comment.char\ =\ ""}, which
means that the comment character is disabled.

\end{frame}

\begin{frame}[fragile]{Getting started}
\protect\hypertarget{getting-started}{}

\begin{itemize}
\item
  Check your working directory
\item
  Change it if you want
\item
  Put the \texttt{mt.csv} file in your current working directory
\item
  Now read it and save the table in an R object named \texttt{mt}
\item
  Open the \texttt{mt} data frame using your global environment or with
  \texttt{View()}
\end{itemize}

\begin{Shaded}
\begin{Highlighting}[]
\NormalTok{mt <-}\StringTok{ }\KeywordTok{read.csv}\NormalTok{(}\DataTypeTok{file =} \StringTok{"mt.csv"}\NormalTok{)}
\end{Highlighting}
\end{Shaded}

\begin{Shaded}
\begin{Highlighting}[]
\KeywordTok{View}\NormalTok{(mt)}
\end{Highlighting}
\end{Shaded}

\end{frame}

\begin{frame}[fragile]{Getting started}
\protect\hypertarget{getting-started-1}{}

\begin{itemize}
\item
  Now create a new folder on your working directory an name it
  ``datasets''.
\item
  Store this lecture's datasets there (except \texttt{mt.csv}).
\end{itemize}

\end{frame}

\begin{frame}[fragile]{Using \texttt{read.csv()}}
\protect\hypertarget{using-read.csv}{}

Read the mtcars0.csv file from the datasets folder and assign it to an
objet:

\begin{Shaded}
\begin{Highlighting}[]
\NormalTok{mtcars0 <-}\StringTok{ }\KeywordTok{read.csv}\NormalTok{(}\DataTypeTok{file =} \StringTok{"datasets/mtcars0.csv"}\NormalTok{)}
\end{Highlighting}
\end{Shaded}

\end{frame}

\begin{frame}[fragile]{Using \texttt{read.csv()}}
\protect\hypertarget{using-read.csv-1}{}

Read the mtcars1.csv file from the datasets folder and assign it to an
object:

\begin{Shaded}
\begin{Highlighting}[]
\NormalTok{mtcars1 <-}\StringTok{ }\KeywordTok{read.csv}\NormalTok{(}\DataTypeTok{file =} \StringTok{"datasets/mtcars1.csv"}\NormalTok{,}
                    \DataTypeTok{row.names =} \DecValTok{1}\NormalTok{)}
\end{Highlighting}
\end{Shaded}

\end{frame}

\begin{frame}[fragile]{Using \texttt{read.csv2()}}
\protect\hypertarget{using-read.csv2}{}

Read the airquality0.csv file from the datasets folder and assign it to
an object:

\begin{Shaded}
\begin{Highlighting}[]
\NormalTok{air0 <-}\StringTok{ }\KeywordTok{read.csv2}\NormalTok{(}\StringTok{"datasets/airquality0.csv"}\NormalTok{)}
\end{Highlighting}
\end{Shaded}

\end{frame}

\begin{frame}[fragile]{Using \texttt{read.csv2()}}
\protect\hypertarget{using-read.csv2-1}{}

Read the airquality1.csv file from the datasets folder and assign it to
an object:

\begin{Shaded}
\begin{Highlighting}[]
\NormalTok{air1 <-}\StringTok{ }\KeywordTok{read.csv2}\NormalTok{(}\StringTok{"datasets/airquality1.csv"}\NormalTok{,}
                  \DataTypeTok{row.names =} \DecValTok{1}\NormalTok{) }
\end{Highlighting}
\end{Shaded}

\end{frame}

\begin{frame}[fragile]{Using \texttt{read.delim()}}
\protect\hypertarget{using-read.delim}{}

Read the iris0.txt file from the datasets folder and assign it to an
object:

\begin{Shaded}
\begin{Highlighting}[]
\NormalTok{iris0 <-}\StringTok{ }\KeywordTok{read.delim}\NormalTok{(}\StringTok{"datasets/iris0.txt"}\NormalTok{)}
\end{Highlighting}
\end{Shaded}

\end{frame}

\begin{frame}[fragile]{Using \texttt{read.delim()}}
\protect\hypertarget{using-read.delim-1}{}

Read the iris1.txt file from the datasets folder and assign it to an
object:

\begin{Shaded}
\begin{Highlighting}[]
\NormalTok{iris1 <-}\StringTok{ }\KeywordTok{read.delim}\NormalTok{(}\StringTok{"datasets/iris1.txt"}\NormalTok{,}
                    \DataTypeTok{header =} \OtherTok{FALSE}\NormalTok{,}
                    \DataTypeTok{row.names =} \DecValTok{1}\NormalTok{)}
\end{Highlighting}
\end{Shaded}

\end{frame}

\begin{frame}[fragile]{Using \texttt{read.delim2()}}
\protect\hypertarget{using-read.delim2}{}

Read the weight file from the datasets folder and assign it to an
object:

\begin{Shaded}
\begin{Highlighting}[]
\NormalTok{weight <-}\StringTok{ }\KeywordTok{read.delim2}\NormalTok{(}\StringTok{"datasets/weight.txt"}\NormalTok{)}
\end{Highlighting}
\end{Shaded}

\end{frame}

\begin{frame}[fragile]{Using \texttt{read.delim2()}}
\protect\hypertarget{using-read.delim2-1}{}

Read the womanNA.txt file from the datasets folder and assign it to an
object:

\begin{Shaded}
\begin{Highlighting}[]
\NormalTok{woman <-}\StringTok{ }\KeywordTok{read.delim2}\NormalTok{(}\StringTok{"datasets/womanNA.txt"}\NormalTok{,}
                      \DataTypeTok{na.strings =} \KeywordTok{c}\NormalTok{(}\StringTok{"NA"}\NormalTok{, }\StringTok{"Na"}\NormalTok{))}
\end{Highlighting}
\end{Shaded}

\end{frame}

\begin{frame}[fragile]{Using \texttt{read.table()}}
\protect\hypertarget{using-read.table}{}

Read the woman.txt file from the datasets folder and assign it to an
object:

\begin{Shaded}
\begin{Highlighting}[]
\NormalTok{woman2 <-}\StringTok{ }\KeywordTok{read.table}\NormalTok{(}\StringTok{"datasets/woman.txt"}\NormalTok{,}
                     \DataTypeTok{sep =} \StringTok{"}\CharTok{\textbackslash{}t}\StringTok{"}\NormalTok{,}
                     \DataTypeTok{header =} \OtherTok{TRUE}\NormalTok{)}
\end{Highlighting}
\end{Shaded}

\end{frame}

\begin{frame}[fragile]{Using \texttt{read.table()}}
\protect\hypertarget{using-read.table-1}{}

Read the AirPassengers.csv file from the datasets folder and assign it
to an object:

\begin{Shaded}
\begin{Highlighting}[]
\NormalTok{airPass <-}\StringTok{ }\KeywordTok{read.table}\NormalTok{(}\StringTok{"datasets/AirPassengers.csv"}\NormalTok{,}
                      \DataTypeTok{sep =} \StringTok{"|"}\NormalTok{)}
\end{Highlighting}
\end{Shaded}

Why didn´t I use \texttt{row.names\ =\ 1}?

\end{frame}

\begin{frame}[fragile]{Using \texttt{read.table()}}
\protect\hypertarget{using-read.table-2}{}

Read the ChickWeight.txt file from the datasets folder and assign it to
an object:

\begin{Shaded}
\begin{Highlighting}[]
\NormalTok{chick <-}\StringTok{ }\KeywordTok{read.table}\NormalTok{(}\StringTok{"datasets/ChickWeight.txt"}\NormalTok{,}
                    \DataTypeTok{sep =} \StringTok{"&"}\NormalTok{,}
                    \DataTypeTok{row.names =} \DecValTok{1}\NormalTok{)}
\end{Highlighting}
\end{Shaded}

\end{frame}

\begin{frame}[fragile]{Using \texttt{read.table()}}
\protect\hypertarget{using-read.table-3}{}

Read the iris.csv file from the datasets folder and assign it to an
object:

\begin{Shaded}
\begin{Highlighting}[]
\NormalTok{iris2 <-}\StringTok{ }\KeywordTok{read.table}\NormalTok{(}\StringTok{"datasets/iris.csv"}\NormalTok{,}
                    \DataTypeTok{sep =} \StringTok{";"}\NormalTok{,}
                    \DataTypeTok{dec =} \StringTok{","}\NormalTok{,}
                    \DataTypeTok{header =} \OtherTok{TRUE}\NormalTok{)}
\end{Highlighting}
\end{Shaded}

\end{frame}

\begin{frame}[fragile]{Using \texttt{read.table()}}
\protect\hypertarget{using-read.table-4}{}

Read the irisNA.csv file from the datasets folder and assign it to an
object:

\begin{Shaded}
\begin{Highlighting}[]
\NormalTok{iris3 <-}\StringTok{ }\KeywordTok{read.table}\NormalTok{(}\StringTok{"datasets/irisNA.csv"}\NormalTok{,}
                    \DataTypeTok{sep =} \StringTok{";"}\NormalTok{,}
                    \DataTypeTok{dec =} \StringTok{","}\NormalTok{,}
                    \DataTypeTok{header =} \OtherTok{TRUE}\NormalTok{,}
                    \DataTypeTok{na.strings =} \StringTok{"-"}\NormalTok{)}
\end{Highlighting}
\end{Shaded}

\end{frame}

\begin{frame}[fragile]{Using \texttt{read.table()}}
\protect\hypertarget{using-read.table-5}{}

Read the irisNA2.csv file from the datasets folder and assign it to an
object:

\begin{Shaded}
\begin{Highlighting}[]
\NormalTok{iris4 <-}\StringTok{ }\KeywordTok{read.table}\NormalTok{(}\StringTok{"datasets/irisNA2.csv"}\NormalTok{, }
                    \DataTypeTok{sep =} \StringTok{";"}\NormalTok{,}
                    \DataTypeTok{dec =} \StringTok{","}\NormalTok{,}
                    \DataTypeTok{header =} \OtherTok{TRUE}\NormalTok{,}
                    \DataTypeTok{na.strings =} \StringTok{"-"}\NormalTok{,}
                    \DataTypeTok{skip =} \DecValTok{5}\NormalTok{)}
\end{Highlighting}
\end{Shaded}

\end{frame}

\begin{frame}[fragile]{Using \texttt{read.table()}}
\protect\hypertarget{using-read.table-6}{}

Read the irisNA3.csv file from the datasets folder and assign it to an
object:

\begin{Shaded}
\begin{Highlighting}[]
\NormalTok{iris5 <-}\StringTok{ }\KeywordTok{read.table}\NormalTok{(}\StringTok{"datasets/irisNA3.csv"}\NormalTok{,}
                    \DataTypeTok{sep =} \StringTok{";"}\NormalTok{,}
                    \DataTypeTok{dec =} \StringTok{","}\NormalTok{,}
                    \DataTypeTok{header =} \OtherTok{TRUE}\NormalTok{,}
                    \DataTypeTok{na.strings =} \StringTok{"-"}\NormalTok{,}
                    \DataTypeTok{skip =} \DecValTok{3}\NormalTok{,}
                    \DataTypeTok{comment.char =} \StringTok{"%"}\NormalTok{)}
\end{Highlighting}
\end{Shaded}

\end{frame}

\begin{frame}[fragile]{Reading a file from the internet}
\protect\hypertarget{reading-a-file-from-the-internet}{}

It's also possible to use any of the functions that we've just learned
to import files from the web:

\begin{Shaded}
\begin{Highlighting}[]
\NormalTok{my_data <-}\StringTok{ }\KeywordTok{read.delim}\NormalTok{(}
  \StringTok{"http://www.sthda.com/upload/boxplot_format.txt"}\NormalTok{)}

\KeywordTok{head}\NormalTok{(my_data)}
\end{Highlighting}
\end{Shaded}

\begin{verbatim}
##    Nom variable Group
## 1 IND1       10     A
## 2 IND2        7     A
## 3 IND3       20     A
## 4 IND4       14     A
## 5 IND5       14     A
## 6 IND6       12     A
\end{verbatim}

\end{frame}

\begin{frame}[fragile]{Writing data}
\protect\hypertarget{writing-data}{}

After doing calculations with imported data, we usually want to store
tables with our results in a local file.

This can be done with:

\begin{itemize}
\tightlist
\item
  \texttt{write.table()}
\item
  \texttt{write.csv()}
\item
  \texttt{write.csv2()}
\end{itemize}

\end{frame}

\begin{frame}[fragile]{Writing data}
\protect\hypertarget{writing-data-1}{}

\begin{itemize}
\item
  \texttt{write.table()} is like \texttt{write.table()} in reverse.
\item
  Likewise, \texttt{write.csv()} and \texttt{write.csv2()} are wrappers
  around \texttt{write.table()} differing only in the default values.
\end{itemize}

\end{frame}

\begin{frame}[fragile]{Writing data}
\protect\hypertarget{writing-data-2}{}

These are the most important default values of \texttt{write.table()}:

\begin{Shaded}
\begin{Highlighting}[]
\KeywordTok{write.table}\NormalTok{(x, }\DataTypeTok{file =} \StringTok{""}\NormalTok{, }\DataTypeTok{quote =} \OtherTok{TRUE}\NormalTok{, }\DataTypeTok{sep =} \StringTok{" "}\NormalTok{,}
            \DataTypeTok{na =} \StringTok{"NA"}\NormalTok{, }\DataTypeTok{dec =} \StringTok{"."}\NormalTok{, }\DataTypeTok{row.names =} \OtherTok{TRUE}\NormalTok{,}
            \DataTypeTok{col.names =} \OtherTok{TRUE}\NormalTok{)}
\end{Highlighting}
\end{Shaded}

\begin{itemize}
\item
  \texttt{write.table()} prints object x (after converting it to a data
  frame if it is not one nor a matrix) to a local file.
\item
  In the \texttt{file\ argument} you should indicate a character string
  naming the output file (with the respective path, in case you don't
  want to store the file in the working directory).
\item
  The \texttt{quote} argument indicates if character and factor columns
  should be surrounded by double quotes in the output.
\end{itemize}

\end{frame}

\begin{frame}[fragile]{Writing data}
\protect\hypertarget{writing-data-3}{}

\begin{itemize}
\item
  The \texttt{sep} argument stipulates how the values of the table
  should be separated in the output file.
\item
  The \texttt{na} arguments indictes how the \texttt{NA} values should
  be displayed in the output file
\item
  \texttt{row.names} and \texttt{col.names} indicate whether or not row
  names and column names should be printed to the output file.
\end{itemize}

\end{frame}

\begin{frame}[fragile]{Writing data}
\protect\hypertarget{writing-data-4}{}

\begin{itemize}
\item
  \texttt{write.csv} uses \texttt{","} as separator and \texttt{"."} as
  decimal point
\item
  \texttt{write.csv2} uses \texttt{";"} as separator and \texttt{","} as
  decimal point
\end{itemize}

\end{frame}

\begin{frame}{Writing data}
\protect\hypertarget{writing-data-5}{}

By default:

\begin{itemize}
\tightlist
\item
  In your working directory, create a new folder called outputs.
\end{itemize}

\end{frame}

\begin{frame}[fragile]{Writing data}
\protect\hypertarget{writing-data-6}{}

Write the mtcars1 dataframe to a CSV file with row names and column
names.

\begin{Shaded}
\begin{Highlighting}[]
\KeywordTok{write.csv}\NormalTok{(}\DataTypeTok{x =}\NormalTok{ mtcars1,}
          \DataTypeTok{file =} \StringTok{"outputs/mtcars1.csv"}\NormalTok{)}
\end{Highlighting}
\end{Shaded}

Or:

\begin{Shaded}
\begin{Highlighting}[]
\KeywordTok{write.table}\NormalTok{(}\DataTypeTok{x =}\NormalTok{ mtcars1, }
            \DataTypeTok{file =} \StringTok{"outputs/mtcars1.csv"}\NormalTok{,}
            \DataTypeTok{sep =} \StringTok{";"}\NormalTok{)}
\end{Highlighting}
\end{Shaded}

\end{frame}

\begin{frame}[fragile]{Writing data}
\protect\hypertarget{writing-data-7}{}

Write the mtcars1 dataframe to a CSV file with row names, column names,
and values separated with \texttt{";"}. Use \texttt{","} as the decimal
separator.

\begin{Shaded}
\begin{Highlighting}[]
\KeywordTok{write.csv2}\NormalTok{(}\DataTypeTok{x =}\NormalTok{ mtcars1,}
           \DataTypeTok{file =} \StringTok{"outputs/mtcars1_v2.csv"}\NormalTok{)}
\end{Highlighting}
\end{Shaded}

Or:

\begin{Shaded}
\begin{Highlighting}[]
\KeywordTok{write.table}\NormalTok{(}\DataTypeTok{x =}\NormalTok{ mtcars1, }
            \DataTypeTok{file =} \StringTok{"outputs/mtcars1_v2.csv"}\NormalTok{, }
            \DataTypeTok{sep =} \StringTok{";"}\NormalTok{,}
            \DataTypeTok{dec =} \StringTok{","}\NormalTok{)}
\end{Highlighting}
\end{Shaded}

\end{frame}

\begin{frame}[fragile]{Writing data}
\protect\hypertarget{writing-data-8}{}

Write the air0 dataframe to a tab delimited file (.txt). Include column
names but not row names, and identify the NAs with ``-''.

\begin{Shaded}
\begin{Highlighting}[]
\KeywordTok{write.table}\NormalTok{(air0, }\DataTypeTok{file =} \StringTok{"outputs/air0.txt"}\NormalTok{, }
            \DataTypeTok{sep =} \StringTok{"}\CharTok{\textbackslash{}t}\StringTok{"}\NormalTok{,}
            \DataTypeTok{row.names =} \OtherTok{FALSE}\NormalTok{,}
            \DataTypeTok{na =} \StringTok{"-"}\NormalTok{)}
\end{Highlighting}
\end{Shaded}

\end{frame}

\begin{frame}[fragile]{Writing data}
\protect\hypertarget{writing-data-9}{}

Write the chick dataframe to a text file without row names nor column
names. Table values in the output should be separated by ``\textbar{}''.

\begin{Shaded}
\begin{Highlighting}[]
\KeywordTok{write.table}\NormalTok{(chick, }\DataTypeTok{file =} \StringTok{"outputs/chick.txt"}\NormalTok{,}
            \DataTypeTok{sep =} \StringTok{"|"}\NormalTok{,}
            \DataTypeTok{row.names =} \OtherTok{FALSE}\NormalTok{,}
            \DataTypeTok{col.names =} \OtherTok{FALSE}\NormalTok{)}
\end{Highlighting}
\end{Shaded}

\end{frame}

\end{document}
