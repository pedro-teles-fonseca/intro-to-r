\PassOptionsToPackage{unicode=true}{hyperref} % options for packages loaded elsewhere
\PassOptionsToPackage{hyphens}{url}
\PassOptionsToPackage{dvipsnames,svgnames*,x11names*}{xcolor}
%
\documentclass[ignorenonframetext,]{beamer}
\usepackage{pgfpages}
\setbeamertemplate{caption}[numbered]
\setbeamertemplate{caption label separator}{: }
\setbeamercolor{caption name}{fg=normal text.fg}
\beamertemplatenavigationsymbolsempty
% Prevent slide breaks in the middle of a paragraph:
\widowpenalties 1 10000
\raggedbottom
\setbeamertemplate{part page}{
\centering
\begin{beamercolorbox}[sep=16pt,center]{part title}
  \usebeamerfont{part title}\insertpart\par
\end{beamercolorbox}
}
\setbeamertemplate{section page}{
\centering
\begin{beamercolorbox}[sep=12pt,center]{part title}
  \usebeamerfont{section title}\insertsection\par
\end{beamercolorbox}
}
\setbeamertemplate{subsection page}{
\centering
\begin{beamercolorbox}[sep=8pt,center]{part title}
  \usebeamerfont{subsection title}\insertsubsection\par
\end{beamercolorbox}
}
\AtBeginPart{
  \frame{\partpage}
}
\AtBeginSection{
  \ifbibliography
  \else
    \frame{\sectionpage}
  \fi
}
\AtBeginSubsection{
  \frame{\subsectionpage}
}
\usepackage{lmodern}
\usepackage{amssymb,amsmath}
\usepackage{ifxetex,ifluatex}
\usepackage{fixltx2e} % provides \textsubscript
\ifnum 0\ifxetex 1\fi\ifluatex 1\fi=0 % if pdftex
  \usepackage[T1]{fontenc}
  \usepackage[utf8]{inputenc}
  \usepackage{textcomp} % provides euro and other symbols
\else % if luatex or xelatex
  \usepackage{unicode-math}
  \defaultfontfeatures{Ligatures=TeX,Scale=MatchLowercase}
\fi
% use upquote if available, for straight quotes in verbatim environments
\IfFileExists{upquote.sty}{\usepackage{upquote}}{}
% use microtype if available
\IfFileExists{microtype.sty}{%
\usepackage[]{microtype}
\UseMicrotypeSet[protrusion]{basicmath} % disable protrusion for tt fonts
}{}
\IfFileExists{parskip.sty}{%
\usepackage{parskip}
}{% else
\setlength{\parindent}{0pt}
\setlength{\parskip}{6pt plus 2pt minus 1pt}
}
\usepackage{xcolor}
\usepackage{hyperref}
\hypersetup{
            pdftitle={Introduction to R Programming},
            pdfauthor={Pedro Fonseca},
            colorlinks=true,
            linkcolor=Maroon,
            filecolor=Maroon,
            citecolor=Blue,
            urlcolor=blue,
            breaklinks=true}
\urlstyle{same}  % don't use monospace font for urls
\newif\ifbibliography
\usepackage{color}
\usepackage{fancyvrb}
\newcommand{\VerbBar}{|}
\newcommand{\VERB}{\Verb[commandchars=\\\{\}]}
\DefineVerbatimEnvironment{Highlighting}{Verbatim}{commandchars=\\\{\}}
% Add ',fontsize=\small' for more characters per line
\usepackage{framed}
\definecolor{shadecolor}{RGB}{248,248,248}
\newenvironment{Shaded}{\begin{snugshade}}{\end{snugshade}}
\newcommand{\AlertTok}[1]{\textcolor[rgb]{0.94,0.16,0.16}{#1}}
\newcommand{\AnnotationTok}[1]{\textcolor[rgb]{0.56,0.35,0.01}{\textbf{\textit{#1}}}}
\newcommand{\AttributeTok}[1]{\textcolor[rgb]{0.77,0.63,0.00}{#1}}
\newcommand{\BaseNTok}[1]{\textcolor[rgb]{0.00,0.00,0.81}{#1}}
\newcommand{\BuiltInTok}[1]{#1}
\newcommand{\CharTok}[1]{\textcolor[rgb]{0.31,0.60,0.02}{#1}}
\newcommand{\CommentTok}[1]{\textcolor[rgb]{0.56,0.35,0.01}{\textit{#1}}}
\newcommand{\CommentVarTok}[1]{\textcolor[rgb]{0.56,0.35,0.01}{\textbf{\textit{#1}}}}
\newcommand{\ConstantTok}[1]{\textcolor[rgb]{0.00,0.00,0.00}{#1}}
\newcommand{\ControlFlowTok}[1]{\textcolor[rgb]{0.13,0.29,0.53}{\textbf{#1}}}
\newcommand{\DataTypeTok}[1]{\textcolor[rgb]{0.13,0.29,0.53}{#1}}
\newcommand{\DecValTok}[1]{\textcolor[rgb]{0.00,0.00,0.81}{#1}}
\newcommand{\DocumentationTok}[1]{\textcolor[rgb]{0.56,0.35,0.01}{\textbf{\textit{#1}}}}
\newcommand{\ErrorTok}[1]{\textcolor[rgb]{0.64,0.00,0.00}{\textbf{#1}}}
\newcommand{\ExtensionTok}[1]{#1}
\newcommand{\FloatTok}[1]{\textcolor[rgb]{0.00,0.00,0.81}{#1}}
\newcommand{\FunctionTok}[1]{\textcolor[rgb]{0.00,0.00,0.00}{#1}}
\newcommand{\ImportTok}[1]{#1}
\newcommand{\InformationTok}[1]{\textcolor[rgb]{0.56,0.35,0.01}{\textbf{\textit{#1}}}}
\newcommand{\KeywordTok}[1]{\textcolor[rgb]{0.13,0.29,0.53}{\textbf{#1}}}
\newcommand{\NormalTok}[1]{#1}
\newcommand{\OperatorTok}[1]{\textcolor[rgb]{0.81,0.36,0.00}{\textbf{#1}}}
\newcommand{\OtherTok}[1]{\textcolor[rgb]{0.56,0.35,0.01}{#1}}
\newcommand{\PreprocessorTok}[1]{\textcolor[rgb]{0.56,0.35,0.01}{\textit{#1}}}
\newcommand{\RegionMarkerTok}[1]{#1}
\newcommand{\SpecialCharTok}[1]{\textcolor[rgb]{0.00,0.00,0.00}{#1}}
\newcommand{\SpecialStringTok}[1]{\textcolor[rgb]{0.31,0.60,0.02}{#1}}
\newcommand{\StringTok}[1]{\textcolor[rgb]{0.31,0.60,0.02}{#1}}
\newcommand{\VariableTok}[1]{\textcolor[rgb]{0.00,0.00,0.00}{#1}}
\newcommand{\VerbatimStringTok}[1]{\textcolor[rgb]{0.31,0.60,0.02}{#1}}
\newcommand{\WarningTok}[1]{\textcolor[rgb]{0.56,0.35,0.01}{\textbf{\textit{#1}}}}
\setlength{\emergencystretch}{3em}  % prevent overfull lines
\providecommand{\tightlist}{%
  \setlength{\itemsep}{0pt}\setlength{\parskip}{0pt}}
\setcounter{secnumdepth}{0}

% set default figure placement to htbp
\makeatletter
\def\fps@figure{htbp}
\makeatother


\title{Introduction to R Programming}
\providecommand{\subtitle}[1]{}
\subtitle{Data Frames}
\author{Pedro Fonseca}
\date{07 Maio 2020}

\begin{document}
\frame{\titlepage}

\begin{frame}[fragile]{The \texttt{c()} Operator}
\protect\hypertarget{the-c-operator}{}

\begin{itemize}
\item
  Vectors are essential building blocks for handling multiple items in
  R.
\item
  To create vectors use the \emph{combine} operator:
\end{itemize}

\begin{Shaded}
\begin{Highlighting}[]
\NormalTok{x <-}\StringTok{  }\DecValTok{-1}

\NormalTok{y <-}\StringTok{  }\DecValTok{10}

\NormalTok{z <-}\StringTok{  }\KeywordTok{c}\NormalTok{(}\DecValTok{1}\NormalTok{, }\DecValTok{6}\NormalTok{, }\DecValTok{3}\NormalTok{)}

\NormalTok{(myvec <-}\StringTok{ }\KeywordTok{c}\NormalTok{(}\DecValTok{1}\NormalTok{, }\DecValTok{3}\NormalTok{, }\DecValTok{1}\NormalTok{, }\DecValTok{42}\NormalTok{))}
\end{Highlighting}
\end{Shaded}

\begin{verbatim}
## [1]  1  3  1 42
\end{verbatim}

\begin{Shaded}
\begin{Highlighting}[]
\NormalTok{(myvec2 <-}\StringTok{ }\KeywordTok{c}\NormalTok{(myvec, x, y ,z))}
\end{Highlighting}
\end{Shaded}

\begin{verbatim}
## [1]  1  3  1 42 -1 10  1  6  3
\end{verbatim}

\begin{Shaded}
\begin{Highlighting}[]
\NormalTok{(myvec3 <-}\StringTok{  }\KeywordTok{c}\NormalTok{(myvec, }\DecValTok{1}\NormalTok{, }\DecValTok{2}\NormalTok{))}
\end{Highlighting}
\end{Shaded}

\begin{verbatim}
## [1]  1  3  1 42  1  2
\end{verbatim}

\end{frame}

\begin{frame}[fragile]{Subsetting}
\protect\hypertarget{subsetting}{}

Get the first element:

\begin{Shaded}
\begin{Highlighting}[]
\NormalTok{myvec[}\DecValTok{1}\NormalTok{]}
\end{Highlighting}
\end{Shaded}

\begin{verbatim}
## [1] 1
\end{verbatim}

Get the second element:

\begin{Shaded}
\begin{Highlighting}[]
\NormalTok{myvec[}\DecValTok{2}\NormalTok{]}
\end{Highlighting}
\end{Shaded}

\begin{verbatim}
## [1] 3
\end{verbatim}

\end{frame}

\begin{frame}[fragile]{Subsetting}
\protect\hypertarget{subsetting-1}{}

Get the first three elements:

\begin{Shaded}
\begin{Highlighting}[]
\NormalTok{myvec[}\DecValTok{1}\OperatorTok{:}\DecValTok{3}\NormalTok{]}
\end{Highlighting}
\end{Shaded}

\begin{verbatim}
## [1] 1 3 1
\end{verbatim}

Omit the first element:

\begin{Shaded}
\begin{Highlighting}[]
\NormalTok{myvec[}\OperatorTok{-}\DecValTok{1}\NormalTok{]}
\end{Highlighting}
\end{Shaded}

\begin{verbatim}
## [1]  3  1 42
\end{verbatim}

Omit more than one element:

\begin{Shaded}
\begin{Highlighting}[]
\NormalTok{myvec[}\OperatorTok{-}\KeywordTok{c}\NormalTok{(}\DecValTok{1}\NormalTok{,}\DecValTok{2}\NormalTok{)]}
\end{Highlighting}
\end{Shaded}

\begin{verbatim}
## [1]  1 42
\end{verbatim}

\end{frame}

\begin{frame}[fragile]{Overwriting}
\protect\hypertarget{overwriting}{}

Substitute an element:

\begin{Shaded}
\begin{Highlighting}[]
\NormalTok{myvec[}\DecValTok{3}\NormalTok{] <-}\StringTok{ }\DecValTok{6}
\NormalTok{myvec}
\end{Highlighting}
\end{Shaded}

\begin{verbatim}
## [1]  1  3  6 42
\end{verbatim}

Substitute more than one element:

\begin{Shaded}
\begin{Highlighting}[]
\NormalTok{myvec[}\KeywordTok{c}\NormalTok{(}\DecValTok{2}\NormalTok{,}\DecValTok{3}\NormalTok{,}\DecValTok{4}\NormalTok{)] <-}\StringTok{ }\KeywordTok{c}\NormalTok{(}\DecValTok{2}\NormalTok{,}\DecValTok{3}\NormalTok{,}\DecValTok{4}\NormalTok{)}
\end{Highlighting}
\end{Shaded}

\end{frame}

\begin{frame}[fragile]{Functions to Generate Vectors}
\protect\hypertarget{functions-to-generate-vectors}{}

\begin{Shaded}
\begin{Highlighting}[]
\DecValTok{1}\OperatorTok{:}\DecValTok{10}
\end{Highlighting}
\end{Shaded}

\begin{verbatim}
##  [1]  1  2  3  4  5  6  7  8  9 10
\end{verbatim}

\begin{Shaded}
\begin{Highlighting}[]
\DecValTok{5}\OperatorTok{:}\DecValTok{1}
\end{Highlighting}
\end{Shaded}

\begin{verbatim}
## [1] 5 4 3 2 1
\end{verbatim}

\begin{Shaded}
\begin{Highlighting}[]
\KeywordTok{seq}\NormalTok{(}\DecValTok{1}\NormalTok{, }\DecValTok{10}\NormalTok{)}
\end{Highlighting}
\end{Shaded}

\begin{verbatim}
##  [1]  1  2  3  4  5  6  7  8  9 10
\end{verbatim}

\begin{Shaded}
\begin{Highlighting}[]
\KeywordTok{seq}\NormalTok{(}\DataTypeTok{from =} \DecValTok{18}\NormalTok{, }\DataTypeTok{to =} \DecValTok{27}\NormalTok{, }\DataTypeTok{by =} \DecValTok{3}\NormalTok{)}
\end{Highlighting}
\end{Shaded}

\begin{verbatim}
## [1] 18 21 24 27
\end{verbatim}

\end{frame}

\begin{frame}[fragile]{Functions to Generate Vectors}
\protect\hypertarget{functions-to-generate-vectors-1}{}

\begin{Shaded}
\begin{Highlighting}[]
\KeywordTok{rep}\NormalTok{(}\DataTypeTok{x =} \DecValTok{1}\NormalTok{, }\DataTypeTok{times =} \DecValTok{4}\NormalTok{)}
\end{Highlighting}
\end{Shaded}

\begin{verbatim}
## [1] 1 1 1 1
\end{verbatim}

\begin{Shaded}
\begin{Highlighting}[]
\KeywordTok{rep}\NormalTok{(}\KeywordTok{c}\NormalTok{(}\DecValTok{3}\NormalTok{, }\DecValTok{6}\NormalTok{), }\DataTypeTok{times =} \DecValTok{3}\NormalTok{)}
\end{Highlighting}
\end{Shaded}

\begin{verbatim}
## [1] 3 6 3 6 3 6
\end{verbatim}

\begin{Shaded}
\begin{Highlighting}[]
\KeywordTok{rep}\NormalTok{(}\KeywordTok{c}\NormalTok{(}\DecValTok{3}\NormalTok{, }\DecValTok{62}\NormalTok{, }\FloatTok{8.3}\NormalTok{), }\DataTypeTok{each =} \DecValTok{2}\NormalTok{)}
\end{Highlighting}
\end{Shaded}

\begin{verbatim}
## [1]  3.0  3.0 62.0 62.0  8.3  8.3
\end{verbatim}

\begin{Shaded}
\begin{Highlighting}[]
\KeywordTok{rep}\NormalTok{(}\KeywordTok{c}\NormalTok{(}\DecValTok{3}\NormalTok{, }\DecValTok{6}\NormalTok{), }\DataTypeTok{times =} \DecValTok{3}\NormalTok{, }\DataTypeTok{each =} \DecValTok{2}\NormalTok{)}
\end{Highlighting}
\end{Shaded}

\begin{verbatim}
##  [1] 3 3 6 6 3 3 6 6 3 3 6 6
\end{verbatim}

\end{frame}

\begin{frame}[fragile]{Sorting the Elements of a Vector}
\protect\hypertarget{sorting-the-elements-of-a-vector}{}

Sorting a vector in increasing or decreasing order:

\begin{Shaded}
\begin{Highlighting}[]
\NormalTok{myvec2 <-}\StringTok{ }\KeywordTok{c}\NormalTok{(}\DecValTok{1}\NormalTok{, }\DecValTok{3}\NormalTok{, }\DecValTok{1}\NormalTok{, }\DecValTok{42}\NormalTok{, }\DecValTok{-5}\NormalTok{, }\DecValTok{10}\NormalTok{, }\DecValTok{-50}\NormalTok{)}
\NormalTok{myvec2}
\end{Highlighting}
\end{Shaded}

\begin{verbatim}
## [1]   1   3   1  42  -5  10 -50
\end{verbatim}

\begin{Shaded}
\begin{Highlighting}[]
\KeywordTok{sort}\NormalTok{(myvec2)}
\end{Highlighting}
\end{Shaded}

\begin{verbatim}
## [1] -50  -5   1   1   3  10  42
\end{verbatim}

\begin{Shaded}
\begin{Highlighting}[]
\KeywordTok{sort}\NormalTok{(myvec2, }\DataTypeTok{decreasing =} \OtherTok{TRUE}\NormalTok{)}
\end{Highlighting}
\end{Shaded}

\begin{verbatim}
## [1]  42  10   3   1   1  -5 -50
\end{verbatim}

\end{frame}

\begin{frame}[fragile]{Sorting the Elements of a Vector}
\protect\hypertarget{sorting-the-elements-of-a-vector-1}{}

\begin{Shaded}
\begin{Highlighting}[]
\KeywordTok{sort}\NormalTok{(}\KeywordTok{c}\NormalTok{(}\FloatTok{2.5}\NormalTok{, }\DecValTok{-1}\NormalTok{, }\DecValTok{-10}\NormalTok{, }\FloatTok{3.44}\NormalTok{))}
\end{Highlighting}
\end{Shaded}

\begin{verbatim}
## [1] -10.00  -1.00   2.50   3.44
\end{verbatim}

\begin{Shaded}
\begin{Highlighting}[]
\KeywordTok{sort}\NormalTok{(}\KeywordTok{c}\NormalTok{(}\FloatTok{2.5}\NormalTok{,}\OperatorTok{-}\DecValTok{1}\NormalTok{,}\OperatorTok{-}\DecValTok{10}\NormalTok{,}\FloatTok{3.44}\NormalTok{), }\DataTypeTok{decreasing =} \OtherTok{TRUE}\NormalTok{)}
\end{Highlighting}
\end{Shaded}

\begin{verbatim}
## [1]   3.44   2.50  -1.00 -10.00
\end{verbatim}

\begin{Shaded}
\begin{Highlighting}[]
\KeywordTok{sort}\NormalTok{(}\KeywordTok{c}\NormalTok{(}\FloatTok{2.5}\NormalTok{,}\OperatorTok{-}\DecValTok{1}\NormalTok{,}\OperatorTok{-}\DecValTok{10}\NormalTok{,}\FloatTok{3.44}\NormalTok{), }\OtherTok{TRUE}\NormalTok{)}
\end{Highlighting}
\end{Shaded}

\begin{verbatim}
## [1]   3.44   2.50  -1.00 -10.00
\end{verbatim}

\end{frame}

\begin{frame}[fragile]{Some Statistical Functions}
\protect\hypertarget{some-statistical-functions}{}

\texttt{rnorm(n)} generates \texttt{n} pseudo-random numbers from a
normal distribution (default: \(\mu = 0\), \(\sigma = 1\))

\begin{Shaded}
\begin{Highlighting}[]
\KeywordTok{rnorm}\NormalTok{(}\DecValTok{3}\NormalTok{)}
\end{Highlighting}
\end{Shaded}

\begin{verbatim}
## [1] -1.2067290 -0.6803377 -0.2182001
\end{verbatim}

\begin{Shaded}
\begin{Highlighting}[]
\KeywordTok{rnorm}\NormalTok{(}\DecValTok{4}\NormalTok{, }\DataTypeTok{mean =}  \DecValTok{5}\NormalTok{, }\DataTypeTok{sd =} \DecValTok{2}\NormalTok{)}
\end{Highlighting}
\end{Shaded}

\begin{verbatim}
## [1] 4.841469 5.639049 7.480188 6.494767
\end{verbatim}

Other functions related do the normal distribution:

\begin{itemize}
\item
  \texttt{dnorm} (density), \texttt{pnorm} (distribution function),
  \texttt{qnorm} (quantile function).
\item
  Equivalent functions are available for the most commonly used
  probability distributions: F, t-student, Uniform, Poisson\ldots{}
\end{itemize}

\end{frame}

\begin{frame}[fragile]{The \texttt{set.seed} Function}
\protect\hypertarget{the-set.seed-function}{}

\begin{itemize}
\item
  Functions like \texttt{rnorm}, \texttt{rpois} and \texttt{runif}
  generate pseudo-random numbers.
\item
  This means that you and I will get different results when using these
  functions.
\item
  Solution: use the \texttt{set.seed} function.
\end{itemize}

\end{frame}

\begin{frame}[fragile]{The \texttt{set.seed} Function}
\protect\hypertarget{the-set.seed-function-1}{}

Try this command many times:

\begin{Shaded}
\begin{Highlighting}[]
\KeywordTok{rnorm}\NormalTok{(}\DecValTok{2}\NormalTok{) }
\end{Highlighting}
\end{Shaded}

\begin{verbatim}
## [1]  1.0358944 -0.7403707
\end{verbatim}

Each time you will get a different output. Now try this:

\begin{Shaded}
\begin{Highlighting}[]
\KeywordTok{set.seed}\NormalTok{(}\DecValTok{123}\NormalTok{)}
\KeywordTok{rnorm}\NormalTok{(}\DecValTok{2}\NormalTok{)}
\end{Highlighting}
\end{Shaded}

\begin{verbatim}
## [1] -0.5604756 -0.2301775
\end{verbatim}

You will get the same output every time. The argument of
\textit{set.seed} is irrelevant as long as we all use the same value.

\end{frame}

\begin{frame}{Vectorized Operations}
\protect\hypertarget{vectorized-operations}{}

One of the main advantages of R is vectorized calculation. This means
that:

\begin{itemize}
\item
  Most R functions accept vectors as inputs;
\item
  Vector arithmetic is performed element-wise by default.
\end{itemize}

\end{frame}

\begin{frame}{Vectorized Operations}
\protect\hypertarget{vectorized-operations-1}{}

\begin{itemize}
\item
  Vectorization calculation is a huge advantage efficiency and
  parsimony.
\item
  Vectorization also makes code easier to write and read.
\end{itemize}

\end{frame}

\begin{frame}[fragile]{Vectorized Operations}
\protect\hypertarget{vectorized-operations-2}{}

\begin{Shaded}
\begin{Highlighting}[]
\NormalTok{x <-}\StringTok{ }\KeywordTok{c}\NormalTok{(}\DecValTok{1}\NormalTok{, }\DecValTok{2}\NormalTok{, }\DecValTok{3}\NormalTok{)}
\NormalTok{y <-}\StringTok{ }\KeywordTok{c}\NormalTok{(}\FloatTok{0.5}\NormalTok{, }\FloatTok{0.5}\NormalTok{, }\FloatTok{0.5}\NormalTok{)}
\DecValTok{1}\OperatorTok{/}\NormalTok{x}
\end{Highlighting}
\end{Shaded}

\begin{verbatim}
## [1] 1.0000000 0.5000000 0.3333333
\end{verbatim}

\begin{Shaded}
\begin{Highlighting}[]
\DecValTok{3}\OperatorTok{+}\NormalTok{y}
\end{Highlighting}
\end{Shaded}

\begin{verbatim}
## [1] 3.5 3.5 3.5
\end{verbatim}

\begin{Shaded}
\begin{Highlighting}[]
\NormalTok{x}\OperatorTok{+}\NormalTok{y}
\end{Highlighting}
\end{Shaded}

\begin{verbatim}
## [1] 1.5 2.5 3.5
\end{verbatim}

\end{frame}

\begin{frame}[fragile]{Vectorized Operations}
\protect\hypertarget{vectorized-operations-3}{}

\begin{Shaded}
\begin{Highlighting}[]
\NormalTok{x <-}\StringTok{ }\KeywordTok{c}\NormalTok{(}\DecValTok{1}\NormalTok{, }\DecValTok{2}\NormalTok{, }\DecValTok{3}\NormalTok{)}
\NormalTok{y <-}\StringTok{ }\KeywordTok{c}\NormalTok{(}\FloatTok{0.5}\NormalTok{, }\FloatTok{0.5}\NormalTok{, }\FloatTok{0.5}\NormalTok{)}

\NormalTok{x}\OperatorTok{^}\NormalTok{y}
\end{Highlighting}
\end{Shaded}

\begin{verbatim}
## [1] 1.000000 1.414214 1.732051
\end{verbatim}

\begin{Shaded}
\begin{Highlighting}[]
\KeywordTok{sqrt}\NormalTok{(x)}
\end{Highlighting}
\end{Shaded}

\begin{verbatim}
## [1] 1.000000 1.414214 1.732051
\end{verbatim}

\begin{Shaded}
\begin{Highlighting}[]
\DecValTok{1}\OperatorTok{/}\DecValTok{1}\OperatorTok{:}\DecValTok{3}
\end{Highlighting}
\end{Shaded}

\begin{verbatim}
## [1] 1.0000000 0.5000000 0.3333333
\end{verbatim}

\begin{Shaded}
\begin{Highlighting}[]
\KeywordTok{seq}\NormalTok{(}\DataTypeTok{from =} \DecValTok{2}\NormalTok{, }\DataTypeTok{to =} \DecValTok{6}\NormalTok{, }\DataTypeTok{by =} \DecValTok{2}\NormalTok{)}\OperatorTok{/}\DecValTok{2}
\end{Highlighting}
\end{Shaded}

\begin{verbatim}
## [1] 1 2 3
\end{verbatim}

\end{frame}

\begin{frame}[fragile]{Vectorized Operations}
\protect\hypertarget{vectorized-operations-4}{}

\begin{Shaded}
\begin{Highlighting}[]
\NormalTok{x1 <-}\StringTok{ }\KeywordTok{c}\NormalTok{(}\DecValTok{1}\NormalTok{, }\DecValTok{5}\NormalTok{, }\DecValTok{7}\NormalTok{)}
\NormalTok{x2 <-}\StringTok{ }\KeywordTok{rep}\NormalTok{(}\DecValTok{1}\NormalTok{, }\DataTypeTok{times =} \DecValTok{3}\NormalTok{)}

\KeywordTok{log}\NormalTok{(x1)}
\end{Highlighting}
\end{Shaded}

\begin{verbatim}
## [1] 0.000000 1.609438 1.945910
\end{verbatim}

\begin{Shaded}
\begin{Highlighting}[]
\KeywordTok{log}\NormalTok{(x1) }\OperatorTok{-}\StringTok{ }\NormalTok{x2}
\end{Highlighting}
\end{Shaded}

\begin{verbatim}
## [1] -1.0000000  0.6094379  0.9459101
\end{verbatim}

\begin{Shaded}
\begin{Highlighting}[]
\NormalTok{x <-}\StringTok{ }\NormalTok{x1 }\OperatorTok{+}\StringTok{ }\NormalTok{x2}
\NormalTok{x}
\end{Highlighting}
\end{Shaded}

\begin{verbatim}
## [1] 2 6 8
\end{verbatim}

\end{frame}

\begin{frame}[fragile]{Rounding}
\protect\hypertarget{rounding}{}

\texttt{round()} rounds the values in its first argument to the
specified number of decimal places (default 0):

\begin{Shaded}
\begin{Highlighting}[]
\KeywordTok{set.seed}\NormalTok{(}\DecValTok{123}\NormalTok{)}

\NormalTok{z <-}\StringTok{ }\KeywordTok{rnorm}\NormalTok{(}\DecValTok{3}\NormalTok{)}
\NormalTok{z}
\end{Highlighting}
\end{Shaded}

\begin{verbatim}
## [1] -0.5604756 -0.2301775  1.5587083
\end{verbatim}

\begin{Shaded}
\begin{Highlighting}[]
\KeywordTok{round}\NormalTok{(z, }\DataTypeTok{digits =} \DecValTok{3}\NormalTok{)}
\end{Highlighting}
\end{Shaded}

\begin{verbatim}
## [1] -0.560 -0.230  1.559
\end{verbatim}

\begin{Shaded}
\begin{Highlighting}[]
\KeywordTok{round}\NormalTok{(z)}
\end{Highlighting}
\end{Shaded}

\begin{verbatim}
## [1] -1  0  2
\end{verbatim}

\end{frame}

\begin{frame}[fragile]{Rounding}
\protect\hypertarget{rounding-1}{}

\begin{Shaded}
\begin{Highlighting}[]
\NormalTok{y <-}\StringTok{ }\KeywordTok{c}\NormalTok{(}\FloatTok{3.271109}\NormalTok{, }\FloatTok{3.374961}\NormalTok{, }\FloatTok{2.313307}\NormalTok{, }\FloatTok{4.837787}\NormalTok{)}
\KeywordTok{round}\NormalTok{(y, }\DecValTok{2}\NormalTok{)}
\end{Highlighting}
\end{Shaded}

\begin{verbatim}
## [1] 3.27 3.37 2.31 4.84
\end{verbatim}

\end{frame}

\begin{frame}[fragile]{Statistical Functions}
\protect\hypertarget{statistical-functions}{}

\begin{Shaded}
\begin{Highlighting}[]
\NormalTok{z}
\end{Highlighting}
\end{Shaded}

\begin{verbatim}
## [1] -0.5604756 -0.2301775  1.5587083
\end{verbatim}

\begin{Shaded}
\begin{Highlighting}[]
\KeywordTok{abs}\NormalTok{(z) }\CommentTok{# Absolut value}
\end{Highlighting}
\end{Shaded}

\begin{verbatim}
## [1] 0.5604756 0.2301775 1.5587083
\end{verbatim}

\begin{Shaded}
\begin{Highlighting}[]
\KeywordTok{max}\NormalTok{(z)}
\end{Highlighting}
\end{Shaded}

\begin{verbatim}
## [1] 1.558708
\end{verbatim}

\begin{Shaded}
\begin{Highlighting}[]
\KeywordTok{min}\NormalTok{(z)}
\end{Highlighting}
\end{Shaded}

\begin{verbatim}
## [1] -0.5604756
\end{verbatim}

\end{frame}

\begin{frame}[fragile]{Statistical Functions}
\protect\hypertarget{statistical-functions-1}{}

\begin{Shaded}
\begin{Highlighting}[]
\NormalTok{z}
\end{Highlighting}
\end{Shaded}

\begin{verbatim}
## [1] -0.5604756 -0.2301775  1.5587083
\end{verbatim}

\begin{Shaded}
\begin{Highlighting}[]
\KeywordTok{mean}\NormalTok{(z)}
\end{Highlighting}
\end{Shaded}

\begin{verbatim}
## [1] 0.2560184
\end{verbatim}

\begin{Shaded}
\begin{Highlighting}[]
\KeywordTok{median}\NormalTok{(z)}
\end{Highlighting}
\end{Shaded}

\begin{verbatim}
## [1] -0.2301775
\end{verbatim}

\begin{Shaded}
\begin{Highlighting}[]
\KeywordTok{sd}\NormalTok{(z)}
\end{Highlighting}
\end{Shaded}

\begin{verbatim}
## [1] 1.140186
\end{verbatim}

\end{frame}

\begin{frame}[fragile]{Statistical Functions}
\protect\hypertarget{statistical-functions-2}{}

\begin{Shaded}
\begin{Highlighting}[]
\NormalTok{z}
\end{Highlighting}
\end{Shaded}

\begin{verbatim}
## [1] -0.5604756 -0.2301775  1.5587083
\end{verbatim}

\begin{Shaded}
\begin{Highlighting}[]
\KeywordTok{var}\NormalTok{(z)}
\end{Highlighting}
\end{Shaded}

\begin{verbatim}
## [1] 1.300025
\end{verbatim}

\begin{Shaded}
\begin{Highlighting}[]
\KeywordTok{sum}\NormalTok{(z)}
\end{Highlighting}
\end{Shaded}

\begin{verbatim}
## [1] 0.7680552
\end{verbatim}

\begin{Shaded}
\begin{Highlighting}[]
\KeywordTok{quantile}\NormalTok{(z, }\FloatTok{0.5}\NormalTok{)}
\end{Highlighting}
\end{Shaded}

\begin{verbatim}
##        50% 
## -0.2301775
\end{verbatim}

\end{frame}

\begin{frame}[fragile]{The \texttt{which()} Function}
\protect\hypertarget{the-which-function}{}

The \texttt{which()} function is useful to find which elements of a
vector that verify a given condition:

\begin{Shaded}
\begin{Highlighting}[]
\KeywordTok{set.seed}\NormalTok{(}\DecValTok{123}\NormalTok{)}
\NormalTok{vec <-}\StringTok{ }\KeywordTok{rnorm}\NormalTok{(}\DataTypeTok{n =} \DecValTok{10}\NormalTok{, }\DataTypeTok{mean =} \DecValTok{2}\NormalTok{, }\DataTypeTok{sd =} \DecValTok{1}\NormalTok{)}
\KeywordTok{round}\NormalTok{(vec, }\DecValTok{2}\NormalTok{)}
\end{Highlighting}
\end{Shaded}

\begin{verbatim}
##  [1] 1.44 1.77 3.56 2.07 2.13 3.72 2.46 0.73 1.31 1.55
\end{verbatim}

\begin{Shaded}
\begin{Highlighting}[]
\NormalTok{(indexes <-}\StringTok{ }\KeywordTok{which}\NormalTok{(vec }\OperatorTok{>}\StringTok{  }\DecValTok{2}\NormalTok{))}
\end{Highlighting}
\end{Shaded}

\begin{verbatim}
## [1] 3 4 5 6 7
\end{verbatim}

\begin{Shaded}
\begin{Highlighting}[]
\KeywordTok{round}\NormalTok{(vec[indexes], }\DecValTok{3}\NormalTok{)}
\end{Highlighting}
\end{Shaded}

\begin{verbatim}
## [1] 3.559 2.071 2.129 3.715 2.461
\end{verbatim}

\end{frame}

\begin{frame}[fragile]{The \texttt{which()} Function}
\protect\hypertarget{the-which-function-1}{}

\begin{Shaded}
\begin{Highlighting}[]
\KeywordTok{set.seed}\NormalTok{(}\DecValTok{123}\NormalTok{)}

\NormalTok{vec2 <-}\StringTok{ }\KeywordTok{rpois}\NormalTok{(}\DataTypeTok{n =} \DecValTok{10}\NormalTok{, }\DataTypeTok{lambda =} \DecValTok{2}\NormalTok{)}
\KeywordTok{which}\NormalTok{(vec2 }\OperatorTok{==}\StringTok{ }\DecValTok{2}\NormalTok{)}
\end{Highlighting}
\end{Shaded}

\begin{verbatim}
## [1]  3  7  9 10
\end{verbatim}

\begin{Shaded}
\begin{Highlighting}[]
\NormalTok{(vec2 <-}\StringTok{ }\KeywordTok{rpois}\NormalTok{(}\DataTypeTok{n =} \DecValTok{10}\NormalTok{, }\DataTypeTok{lambda =} \DecValTok{2}\NormalTok{)) }
\end{Highlighting}
\end{Shaded}

\begin{verbatim}
##  [1] 5 2 3 2 0 4 1 0 1 5
\end{verbatim}

\begin{Shaded}
\begin{Highlighting}[]
\KeywordTok{which}\NormalTok{(vec2 }\OperatorTok{==}\StringTok{ }\DecValTok{2}\NormalTok{)}
\end{Highlighting}
\end{Shaded}

\begin{verbatim}
## [1] 2 4
\end{verbatim}

\end{frame}

\begin{frame}[fragile]{The \texttt{which()} Function}
\protect\hypertarget{the-which-function-2}{}

\begin{Shaded}
\begin{Highlighting}[]
\KeywordTok{set.seed}\NormalTok{(}\DecValTok{123}\NormalTok{)}

\NormalTok{vec2 <-}\StringTok{ }\KeywordTok{rpois}\NormalTok{(}\DataTypeTok{n =} \DecValTok{10}\NormalTok{, }\DataTypeTok{lambda =} \DecValTok{2}\NormalTok{)}
\NormalTok{vec2}
\end{Highlighting}
\end{Shaded}

\begin{verbatim}
##  [1] 1 3 2 4 4 0 2 4 2 2
\end{verbatim}

\begin{Shaded}
\begin{Highlighting}[]
\KeywordTok{max}\NormalTok{(vec2)}
\end{Highlighting}
\end{Shaded}

\begin{verbatim}
## [1] 4
\end{verbatim}

\begin{Shaded}
\begin{Highlighting}[]
\KeywordTok{which}\NormalTok{(vec2 }\OperatorTok{==}\StringTok{ }\KeywordTok{max}\NormalTok{(vec2))}
\end{Highlighting}
\end{Shaded}

\begin{verbatim}
## [1] 4 5 8
\end{verbatim}

\end{frame}

\begin{frame}[fragile]{The \texttt{which()} Function}
\protect\hypertarget{the-which-function-3}{}

The \texttt{which} function gives the positions of the elements of the
vectors that verify the condition, not their values!

\begin{Shaded}
\begin{Highlighting}[]
\KeywordTok{set.seed}\NormalTok{(}\DecValTok{123}\NormalTok{)}

\NormalTok{vec2 <-}\StringTok{ }\KeywordTok{rpois}\NormalTok{(}\DataTypeTok{n =} \DecValTok{10}\NormalTok{, }\DataTypeTok{lambda =} \DecValTok{2}\NormalTok{)}
\NormalTok{vec2}
\end{Highlighting}
\end{Shaded}

\begin{verbatim}
##  [1] 1 3 2 4 4 0 2 4 2 2
\end{verbatim}

What are the actual values of \textit{vec2} (not their positions) that
verify the condition?

\begin{Shaded}
\begin{Highlighting}[]
\NormalTok{vec2[}\KeywordTok{which}\NormalTok{(vec }\OperatorTok{>}\StringTok{ }\DecValTok{1}\NormalTok{)]}
\end{Highlighting}
\end{Shaded}

\begin{verbatim}
## [1] 1 3 2 4 4 0 2 2 2
\end{verbatim}

\end{frame}

\begin{frame}[fragile]{The \texttt{length()} Function}
\protect\hypertarget{the-length-function}{}

\begin{Shaded}
\begin{Highlighting}[]
\KeywordTok{round}\NormalTok{(vec[}\KeywordTok{which}\NormalTok{(vec }\OperatorTok{>}\StringTok{ }\DecValTok{2}\NormalTok{)], }\DecValTok{3}\NormalTok{)}
\end{Highlighting}
\end{Shaded}

\begin{verbatim}
## [1] 3.559 2.071 2.129 3.715 2.461
\end{verbatim}

Use \texttt{length()} to obtain the number of elements in a vector:

\begin{Shaded}
\begin{Highlighting}[]
\KeywordTok{length}\NormalTok{(vec)}
\end{Highlighting}
\end{Shaded}

\begin{verbatim}
## [1] 10
\end{verbatim}

How many elements of \texttt{vec} are greater than 2?

\begin{Shaded}
\begin{Highlighting}[]
\KeywordTok{length}\NormalTok{(}\KeywordTok{which}\NormalTok{(vec }\OperatorTok{>}\StringTok{ }\DecValTok{2}\NormalTok{))}
\end{Highlighting}
\end{Shaded}

\begin{verbatim}
## [1] 5
\end{verbatim}

\end{frame}

\begin{frame}[fragile]{Trigonometric Functions}
\protect\hypertarget{trigonometric-functions}{}

R trigonometric take radians as argument, not degrees:

\begin{itemize}
\tightlist
\item
  \(\operatorname{sin}(\frac{\pi}{2})\):
\end{itemize}

\begin{Shaded}
\begin{Highlighting}[]
\KeywordTok{sin}\NormalTok{(pi}\OperatorTok{/}\DecValTok{2}\NormalTok{)}
\end{Highlighting}
\end{Shaded}

\begin{verbatim}
## [1] 1
\end{verbatim}

\begin{itemize}
\tightlist
\item
  \(\operatorname{cos}(\pi)\):
\end{itemize}

\begin{Shaded}
\begin{Highlighting}[]
\KeywordTok{cos}\NormalTok{(pi)}
\end{Highlighting}
\end{Shaded}

\begin{verbatim}
## [1] -1
\end{verbatim}

\begin{itemize}
\tightlist
\item
  \(\operatorname{tan}(\frac{\pi}{3})\):
\end{itemize}

\begin{Shaded}
\begin{Highlighting}[]
\KeywordTok{tan}\NormalTok{(pi}\OperatorTok{/}\DecValTok{3}\NormalTok{)}
\end{Highlighting}
\end{Shaded}

\begin{verbatim}
## [1] 1.732051
\end{verbatim}

\begin{itemize}
\tightlist
\item
  \(\operatorname{cotangent}(\frac{\pi}{3})\):
\end{itemize}

\begin{Shaded}
\begin{Highlighting}[]
\DecValTok{1}\OperatorTok{/}\KeywordTok{tan}\NormalTok{(pi}\OperatorTok{/}\DecValTok{3}\NormalTok{)}
\end{Highlighting}
\end{Shaded}

\begin{verbatim}
## [1] 0.5773503
\end{verbatim}

\end{frame}

\begin{frame}[fragile]{Trigonometric Functions}
\protect\hypertarget{trigonometric-functions-1}{}

Which value has a cosine = -1?

\begin{Shaded}
\begin{Highlighting}[]
\KeywordTok{acos}\NormalTok{(}\OperatorTok{-}\DecValTok{1}\NormalTok{)}
\end{Highlighting}
\end{Shaded}

\begin{verbatim}
## [1] 3.141593
\end{verbatim}

\begin{Shaded}
\begin{Highlighting}[]
\KeywordTok{cos}\NormalTok{(pi)}
\end{Highlighting}
\end{Shaded}

\begin{verbatim}
## [1] -1
\end{verbatim}

Which value has a tangent = 0.5?

\begin{Shaded}
\begin{Highlighting}[]
\KeywordTok{atan}\NormalTok{(}\FloatTok{0.5}\NormalTok{)}
\end{Highlighting}
\end{Shaded}

\begin{verbatim}
## [1] 0.4636476
\end{verbatim}

\begin{Shaded}
\begin{Highlighting}[]
\KeywordTok{tan}\NormalTok{(}\FloatTok{0.4636476}\NormalTok{)}
\end{Highlighting}
\end{Shaded}

\begin{verbatim}
## [1] 0.5
\end{verbatim}

\end{frame}

\begin{frame}[fragile]{Trigonometric Functions}
\protect\hypertarget{trigonometric-functions-2}{}

Trigonometric functions are also vectorized:

\begin{Shaded}
\begin{Highlighting}[]
\NormalTok{(x <-}\StringTok{ }\KeywordTok{seq}\NormalTok{(}\DataTypeTok{from =} \FloatTok{0.25}\NormalTok{, }\DataTypeTok{to =} \DecValTok{1}\NormalTok{, }\DataTypeTok{by =} \FloatTok{0.25}\NormalTok{))}
\end{Highlighting}
\end{Shaded}

\begin{verbatim}
## [1] 0.25 0.50 0.75 1.00
\end{verbatim}

\begin{Shaded}
\begin{Highlighting}[]
\KeywordTok{cos}\NormalTok{(x)}
\end{Highlighting}
\end{Shaded}

\begin{verbatim}
## [1] 0.9689124 0.8775826 0.7316889 0.5403023
\end{verbatim}

\begin{Shaded}
\begin{Highlighting}[]
\DecValTok{1}\OperatorTok{/}\KeywordTok{tan}\NormalTok{(x) }\CommentTok{# cotangent of x}
\end{Highlighting}
\end{Shaded}

\begin{verbatim}
## [1] 3.9163174 1.8304877 1.0734261 0.6420926
\end{verbatim}

\begin{Shaded}
\begin{Highlighting}[]
\KeywordTok{cos}\NormalTok{(x)}\OperatorTok{/}\KeywordTok{sin}\NormalTok{(x) }\CommentTok{# cotangent of x}
\end{Highlighting}
\end{Shaded}

\begin{verbatim}
## [1] 3.9163174 1.8304877 1.0734261 0.6420926
\end{verbatim}

\end{frame}

\begin{frame}[fragile]{Trigonometric Functions}
\protect\hypertarget{trigonometric-functions-3}{}

R has many more trigonometric functions. Try:

\begin{Shaded}
\begin{Highlighting}[]
\NormalTok{?Trig}
\end{Highlighting}
\end{Shaded}

\end{frame}

\begin{frame}[fragile]{Recycling}
\protect\hypertarget{recycling}{}

What happens when we conduct calculations with two vectors of different
length?

\begin{Shaded}
\begin{Highlighting}[]
\NormalTok{myvec <-}\StringTok{ }\KeywordTok{c}\NormalTok{(}\DecValTok{1}\NormalTok{, }\DecValTok{2}\NormalTok{, }\DecValTok{3}\NormalTok{, }\DecValTok{4}\NormalTok{) }
\NormalTok{myvec2 <-}\StringTok{ }\KeywordTok{rep}\NormalTok{(}\FloatTok{0.5}\NormalTok{, }\DataTypeTok{times =} \DecValTok{8}\NormalTok{)}

\NormalTok{myvec }\OperatorTok{+}\StringTok{ }\NormalTok{myvec2}
\end{Highlighting}
\end{Shaded}

\begin{verbatim}
## [1] 1.5 2.5 3.5 4.5 1.5 2.5 3.5 4.5
\end{verbatim}

\end{frame}

\begin{frame}[fragile]{Recycling}
\protect\hypertarget{recycling-1}{}

\begin{Shaded}
\begin{Highlighting}[]
\NormalTok{myvec3 <-}\StringTok{ }\KeywordTok{rep}\NormalTok{(}\FloatTok{0.5}\NormalTok{, }\DataTypeTok{times =} \DecValTok{7}\NormalTok{)}

\NormalTok{myvec }\OperatorTok{+}\StringTok{ }\NormalTok{myvec3}
\end{Highlighting}
\end{Shaded}

\begin{verbatim}
## Warning in myvec + myvec3: longer object length is not a multiple of shorter
## object length
\end{verbatim}

\begin{verbatim}
## [1] 1.5 2.5 3.5 4.5 1.5 2.5 3.5
\end{verbatim}

\end{frame}

\begin{frame}{Recycling}
\protect\hypertarget{recycling-2}{}

\begin{itemize}
\item
  When conducting operations that require input vectors to be of the
  same length, R automatically recycles, or repeats, the shorter one,
  until it is long enough to match the longer one.
\item
  It will only throw an error message if the length of the shorter
  vector is not a multiple of the vector of the larger vector.
\end{itemize}

\end{frame}

\begin{frame}[fragile]{Named Vectors}
\protect\hypertarget{named-vectors}{}

We can also name the elements of a vector:

\begin{Shaded}
\begin{Highlighting}[]
\NormalTok{x <-}\StringTok{ }\KeywordTok{c}\NormalTok{(}\DataTypeTok{x1 =} \DecValTok{1}\NormalTok{, }\DataTypeTok{x2 =}  \DecValTok{4}\NormalTok{, }\DataTypeTok{x3 =} \DecValTok{7}\NormalTok{)}
\NormalTok{x}
\end{Highlighting}
\end{Shaded}

\begin{verbatim}
## x1 x2 x3 
##  1  4  7
\end{verbatim}

Get the names of a vector:

\begin{Shaded}
\begin{Highlighting}[]
\KeywordTok{names}\NormalTok{(x)}
\end{Highlighting}
\end{Shaded}

\begin{verbatim}
## [1] "x1" "x2" "x3"
\end{verbatim}

\end{frame}

\begin{frame}[fragile]{Named Vectors}
\protect\hypertarget{named-vectors-1}{}

The \texttt{names()} function can also be used to provide names to a
vector:

\begin{Shaded}
\begin{Highlighting}[]
\NormalTok{y <-}\StringTok{ }\DecValTok{1}\OperatorTok{:}\DecValTok{3}
\KeywordTok{names}\NormalTok{(y) <-}\StringTok{ }\KeywordTok{c}\NormalTok{(}\StringTok{"y1"}\NormalTok{, }\StringTok{"y2"}\NormalTok{, }\StringTok{"y3"}\NormalTok{)}
\end{Highlighting}
\end{Shaded}

\end{frame}

\begin{frame}[fragile]{Subsetting Named Vectors}
\protect\hypertarget{subsetting-named-vectors}{}

Vectors can also be subseted by name:

\begin{Shaded}
\begin{Highlighting}[]
\NormalTok{y}
\end{Highlighting}
\end{Shaded}

\begin{verbatim}
## y1 y2 y3 
##  1  2  3
\end{verbatim}

\begin{Shaded}
\begin{Highlighting}[]
\NormalTok{y[}\StringTok{"y1"}\NormalTok{]}
\end{Highlighting}
\end{Shaded}

\begin{verbatim}
## y1 
##  1
\end{verbatim}

\begin{Shaded}
\begin{Highlighting}[]
\NormalTok{y[}\KeywordTok{c}\NormalTok{(}\StringTok{"y1"}\NormalTok{, }\StringTok{"y3"}\NormalTok{)]}
\end{Highlighting}
\end{Shaded}

\begin{verbatim}
## y1 y3 
##  1  3
\end{verbatim}

\end{frame}

\begin{frame}[fragile]{The \texttt{paste()} and \texttt{paste0()}
functions}
\protect\hypertarget{the-paste-and-paste0-functions}{}

\texttt{paste()} and \texttt{paste0()} can be useful to generate vector
names:

\begin{Shaded}
\begin{Highlighting}[]
\KeywordTok{paste}\NormalTok{(}\StringTok{"y"}\NormalTok{, }\DecValTok{1}\OperatorTok{:}\KeywordTok{length}\NormalTok{(y), }\DataTypeTok{sep =} \StringTok{""}\NormalTok{)}
\end{Highlighting}
\end{Shaded}

\begin{verbatim}
## [1] "y1" "y2" "y3"
\end{verbatim}

\begin{Shaded}
\begin{Highlighting}[]
\KeywordTok{paste}\NormalTok{(}\StringTok{"name"}\NormalTok{, }\DecValTok{1}\OperatorTok{:}\KeywordTok{length}\NormalTok{(y), }\DataTypeTok{sep =} \StringTok{"_"}\NormalTok{)}
\end{Highlighting}
\end{Shaded}

\begin{verbatim}
## [1] "name_1" "name_2" "name_3"
\end{verbatim}

\begin{Shaded}
\begin{Highlighting}[]
\KeywordTok{paste}\NormalTok{(}\StringTok{"year"}\NormalTok{, }\DecValTok{1990}\OperatorTok{:}\DecValTok{1993}\NormalTok{, }\DataTypeTok{sep =} \StringTok{"-"}\NormalTok{)}
\end{Highlighting}
\end{Shaded}

\begin{verbatim}
## [1] "year-1990" "year-1991" "year-1992" "year-1993"
\end{verbatim}

\begin{Shaded}
\begin{Highlighting}[]
\KeywordTok{paste0}\NormalTok{(}\StringTok{"X"}\NormalTok{, }\DecValTok{1}\OperatorTok{:}\DecValTok{5}\NormalTok{)}
\end{Highlighting}
\end{Shaded}

\begin{verbatim}
## [1] "X1" "X2" "X3" "X4" "X5"
\end{verbatim}

\end{frame}

\end{document}
